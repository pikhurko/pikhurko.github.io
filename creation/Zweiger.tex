\documentclass[12pt]{article}

\usepackage[german]{babel} 

\begin{document}



\title{Imitator der Zukunft\\
 (Ein H"orspiel)}

\author{Oleg Pikhurko}

\date{} % {July 15, 2001}
% revision of October 3, 2002, Sabine's remarks

\maketitle


%Personenlichkeit:
% Professor Peter Harris
% Herr Sam Smilley


SMILLEY. Seien Sie gegr"u"st, Professor Harris. Mein Name ist Sam
Smilley. Ich arbeite bei der Koordinationabteilung.

HARRIS. Guten Tag, Herr Smilley. Ich freue mich sehr, Sie kennezulernen.

SMILLEY. Nennen Sie mich einfach Sam. Ich soll Ihnen zeigen, wie man den
Imitator benutzt.

HARRIS. Entschuldigen Sie bitte. Meinen Sie das Ger"at, das gestern in
mein Gehirn eingepflanzt wurde?

SMILLEY. Nicht genau. Dies, was Sie ein ``Ger"at'' nennen, ist einfach
Interface, das hei"st, das Verbindungsteil zwischen Ihnen und dem
Imitator. Zweiseitige "Ubertragung des Signals; im Prinzip nichts
Kompliziertes. Der Imitator aber ist f"ur Sie, einen Menschen des
Zwanzigsten Jahrhunderts, ein paradoxeres und schwierig zu
begriffendes Ding. Also versuche ich, ihn in vereinfachter Weise zu
erkl"aren und durch einige Ihnen verst"andliche Beispiele zu
veranschaulichen.

HARRIS. Herr Smilley. Ich bin einverstanden, dass ich viele f"ur Sie
elementare Sachen nicht wei"s. Aber wof"ur sind diese Requisiten?
Vielleicht stammen all diese M"obel aus irgendeinem Ihrer Museen, aus
der Ausstellung ``Typische Wohnung der zweiten H"alfte des Zwanzigsten
Jahrhunderts.'' Ich glaube nicht daran, dass dieses Zimmer auch nur
entfernte "Ahnlichkeit mit Ihrer Wohnung hat. Warum haben Sie Angst,
mir die ganze Wahrheit zu zeigen? K"onnen Sie sich vorstellen, was mich
meine Entscheidung gekostet hat, meine Zeit zu verlassen, alles, was
ich hatte, zu verlieren? Die "Uberlebenchance war ganz unbekannt. Was
mich erwarten w"urde, war ganz unbekannt. Ich bin bereit f"ur
alles. F"urchten Sie sich nicht, mich mit Ihrer Realit"at zu
ersch"uttern! H"atten wir irgendeinen Wikinger wiederaufleben
gelassen, h"atten wir ihm alles gezeigt, von New-Yorks Wolkenkratzern
bis Saharas D"unen.

SMILLEY. H"atten Sie Ihrem Wikinger die Wellen- und
Teilcheneigenschaften des Elektrons demonstrieren k"onnen? Diese sind
auch eine Realit"at.

HARRIS. Aber ich bin Wissenschaftler.

SMILLEY. Und? Ihr Wikinger k"onnte auch ein Wissenschaftler seiner Zeit
sein, ein gro"ser Seefahrer, zum Beispiel. H"atte das geholfen?

HARRIS. Doch, er k"onnte ein Flugzeug ber"uhren und sehen, wie es
fliegt, trotzdem h"atte er keine Ahnung, wie das geschieht.

SMILLEY. Falls Sie das meinen, was Ihre Philosophen ``wesentliche
Welt'' nannten, so schafft bei uns jeder (wie soll ich es klarer
sagen?) seine eigene Welt. Auch Sie, obwohl Sie sich dessen nicht
bewusst sind. Meine Welt existiert auch, aber Sie haben noch keine
entsprechenden Sinnesorgane, sie zu untersuchen, h"atte ich Ihnen den
Eintritt erlaubt. Falls Sie die Wunder der Technik sehen m"ochten,
dann w"urde ich Ihnen zeigen, wie unser Imitator fliegen kann.

HARRIS. Gut. Verzeihen Sie mir. Ich wollte Sie nur bitten, wenn es
m"oglich ist, mir auch das Arbeitsprinzip des Imitators ohne
unn"otige Vereinfachungen zu erkl"aren.

SMILLEY. Ich verzeihe Ihnen. Aber verstehen Sie, man h"atte den
Wikinger, einen Fernsehapparat zu ben"utzen, lehren k"onnen, aber wie
kann man ihm die Grundlagen erkl"aren? Na ja, seien Sie nicht
zornig. Ich will Ihnen einen kleinen Trick zeigen. In der rechten
Tasche Ihrer Hosen haben Sie die Zehncent-M"unze. Nehmen Sie sie
heraus.

HARRIS. Aber woher wissen Sie das?

SMILLEY. Sehen Sie mich nicht so verwirrt an, das ist noch kein
Trick. Ziehen die M"unze heraus. Verstecken Sie sie hinter Ihren
R"ucken in irgendeiner Hand... Aber so, dass ich nichts sehe! Nehmen Sie
die H"ande nach vorne!  Also, die M"unze ist in Ihrer linken Hand!

HARRIS. Richtig vermutet. Und?

SMILLEY. Machen wir den Trick noch einmal! Ja. Gut. Die linke!

HARRIS. Richtig.

SMILLEY. Noch einmal!.. Jetzt, die rechte! Genau im Ziel! Wieder! Und jetzt
die linke! Habe getroffen! M"ochten Sie noch einmal?

HARRIS. [nach einiger Pause] Gut, machen wir es noch einmal.

SMILLEY. Bitte. In welcher Hand? Mm... Die M"unze ist in... der
hinteren Tasche Ihrer Hose, wohin Sie sie unbemerkt abgelegt haben!
He-he. Na ja, geben Sie mir die M"unze, Trickbetr"uger!

HARRIS. Und ist dies alles Ihr Imitator?

SMILLEY. Ja. Mein Ehrewort: keine versteckten Kameras, Hypnose, oder
Gedankenlesen! Das ist jedoch nicht alles! Achtung, bitte! Ich werfe
die M"unze. Wappen! Sehen Sie? Ich werfe wieder. Wieder Wappen! Noch
einmal. Sehen Sie, Wappen! Nochmals. Noch einmal Wappen! Und noch mal!

HARRIS. Und was?

SMILLEY. Wundert Sie das nicht?

HARRIS. Nicht besonders.

SMILLEY. Also, Mensch. Nehmen Sie Ihre M"unze zur"uck. Nun will ich
Ihnen zeigen, wie man mit Hilfe des Imitators sogar noch viel
beindruckendere Dinge tun kann.

HARRIS. Wenn m"oglich erkl"aren Sie bitte auch die Grundlagen.

SMILLEY. Machen Sie sich keine Sorgen, ich habe das vermutet und eine
kleine Vorlesung "uber den Imitator vorbereitet. Fangen wir damit an,
dass unsere Zukunft noch nicht fest bestimmt ist. Setzen wir voraus,
dass jemand die vollkommene Information "uber all die Teilchen
unseres Universums in einem einzelnen Augenblick versammelt hat und
ein paar identische unabh"angige Kopien des Universums gemacht
hat. Also, diese Kopien, die zuerst dieselben waren, w"urden sich
andersgeartet entwickeln. Und dies nicht wegen eines Fehlers
unseres Versuchs, sondern das ist das Erscheinen eines tiefen
fraglosen Gesetzes. Stimmen Sie zu?

HARRIS. Ja, es ist wie in der Quantenmechanik. Ich wei"s nicht, wie es
jetzt ist, aber wir glaubten, dass die ganze Welt aus elementaren
Teilchen besteht, jedes von ihnen hat eine bestimmte Wahl. Mit einer
Wahrscheinlichkeit, die von verschiedenen Feldern bestimmt wird, kann
ein Elektron im n"achsten Zeitquant seine Energieniveau erh"ohen oder
vermindern, in dem Fotonen geschluckt oder emittiert werden. Das
Elektron kann, zum Beispiel, auch mit einem Proton zusammensto"sen,
und dabei ein Neutron bilden. Allgemein leite ich daran ab, dass ein
Elektron viele M"oglichkeiten hat, jede passiert mit einer bestimmten
Wahrscheinlichkeit.

SMILLEY. Bravo, ein sehr guter Vergleich. Falls Sie sich daf"ur
interessieren, Quantenmechanik war ein gutes Werkzeug f"ur praktische
Berechnungen von Mikrosystemen, aber sie konnte nicht als eine
Theorie erkl"aren, warum alles so geschieht. Damit besch"aftigt sich
die Allgemeine Feldtheorie, aber Sie w"urden das noch nicht
verstehen.

HARRIS. Ok, zumindest danke f"ur diese kurze Erkl"arung.

SMILLEY. Ich freue mich, dass Sie so leicht mit der Entschlossenheit
der Welt "ubereingestimmt haben. Sie Kennen selbstverst"andlich
Eisenbergs Unbestimmtheitsprinzip?

HARRIS. Dass es unm"oglich ist, auch nur ein Teilchen vollkommen zu
beschreiben?

SMILLEY. Genau, geschweige denn das ganze Universum. Das hei"st, all meine
Gedankenexperimente sind ein solcher Quatsch, dass jeder moderne
Physiker mich sogleich gelyncht h"atte. Aber, erstens erkl"art dieser
Unsinn in gewisser Weisee die echte Wirklichkeit (obwohl sehr
vereinfacht). Zweitens appelliert er an die praktische Erfahrung
eines Menschen, der aus dem Zwangzigsten Jahrhundert stammt und keine
Ahnung "uber den Imitator hat. Oh, entschuldigen Sie bitte, ich habe
wieder angefangen.

HARRIS. Das macht nichts, ich habe mich schon an vereinfachte Modelle
gew"ohnt. Jetzt bin ich eben sehr kritish.

SMILLEY. Wunderbar, Professor. Also, im n"achsten Gedankenexperiment
beschreibt unser allm"achtiger ``jemand'' alle m"oglichen Lagen des
Universums und zeichnet einen Pfeil von Lage A nach Lage B, wenn das
Universum in einem Zeitquant von A nach B wechseln k"onnte. Merken
Sie, jetzt gibt es keine Bewegung, genau wie ein Philosoph gesagt
hat! Alles was es gibt, ist nur eine Menge von Lagen und Pfeilen!

HARRIS. Was bewegt sich dann?

SMILLEY. Zeit! Das, was Zeit hei"st, bewegt sich! Gewiss nicht durch alle
Lagen, sondern an einem zuf"alligen Pfeilpfad entlang, den man
romantischer Weise (und ganz dummerweise) den Zeitfluss nennt.

HARRIS. Das hei"st, es gibt nichts  ausserhalb vor dem Zeitfluss?

SMILLEY. Falsch. Dass wir Lagen ausserhalb des Zeitflusses nicht
beobachten (oder, um es genauer zu sagen, nicht direkt beobachten),
bedeutet nicht, dass sie nicht existieren.

HARRIS. Gut, ich k"onnte zustimmen.

SMILLEY. Unsere Welt ist zuf"allig, trotzdem sind ``fast Regeln'' da.
Wenn ich diese M"unze lasse, f"allt sie zu Boden. Wenn ich einen
schweren Ziegelstein auf den Tisch lege, bleibt er weiterhin
Ziegelstein und wird auf demselben Platz liegenbleiben. Mit
vollkommener Sicherheit wechselt er pl"otzlich weder seine Form noch
seinen Standort. Und das gilt nicht nur f"ur die Lagen auf dem
Zeitfluss, sondern auch f"ur die neunundneunzig Komme mit einer
Neunherde Prozenten von allen anderen Lagen, die der Zeitfluss h"atte
erreichen k"onnen.

HARRIS. Das ist klar: obwohl sich die einzelnen Teilchen zuf"allig
bewegen, kompensieren sich diese Unregelm"a"sigkeiten, wenn man die
Gesamtmenge betrachtet. Richtig?

SMILLEY. Genau, aber betrachten wir Lagen au"serfalls des Zeitflusses,
die man `virtuell' nennt im Gegenteil zu den `realen' Lagen auf dem
Zeitfluss. Wie k"onnte man sie beobachten? Dazu brauchen wir etwas,
was zur"uckkehrt. Zum Gl"uck, dieses `etwas' existiert: die
Antimaterie! Aber, da die Materie in unserem Lagengraph bewegungslos
ist, soll es etwas geben, was die Antimateriebewegung
repr"asentiert. Und, wie einfach zu vermuten ist, hei"st dieses die
`Antizeit'.

HARRIS. Ja, wahrscheinlich.

SMILLEY. Nehmen wir an, dass ich eine wichtige und schwere Wahl
habe. Zum Beispiel, ``Ist die M"unze in Peters rechter oder linker
Hand?''  He-he! Was mache ich? Ich sage ``Rechte!'' Wunderbar, wenn
das richtig ist. Falls nicht, dann lege ich, wie ein echter Gentleman,
einen Antiziegel auf den Tisch, auf dem ``Dummkopf, nicht die Rechte!''
geschrieben ist. Und gleich kommt das Wichtigste! Bevor ich
``Rechte!'' rufe, probiere ich mit der Hand, ob der Ziegel auf dem
Tisch liegt. Und, falls er da ist, proklamiere ich ``Linke!'', sonst
``Rechte!''  Alles klar?

HARRIS. Warten Sie, ich muss "uberlegen... Aha, es ist so... Doch!
Woher wissen Sie, dass der Antiziegel durch denselben Pfad
zur"uckkehrt?

SMILLEY. Er macht das gewiss nicht! Die Zeit- und Antizeitfl"usse sind
nicht dieselben. Aber! Wenn ich einen ordentlichen Ziegel in Lage A
auf den Tisch lege, dann "uberlebt er (und man kann ihn entdecken) in
99 (ich will die ganze Neune nicht wiederholen) Prozent der Lagen,
reale oder virtuelle, die von A, sagen wir in einer Stunde, erreichbar
sind. Gleicherweise, wenn ich einen Antiziegel in Lage A lasse, kann
man ihn in 99 Prozent der Lagen entdecken, von denen man Lage A
erreichen kann. "Uberenken Sie das.

HARRIS. In Ordnung... Aber dann stimmt die Ursache-Konsequenz Kette nicht!

SMILLEY. Doch, alles stimmt! Weil das ersten Experiment, wenn ich den
Antiziegel nicht suche, au"serfalls des Zeitflusses ist! Sehen Sie,
wenn ich meine Wahl auf die An- oder Abwesenheit des Antiziegels
gr"unde, existiert die Situation, in der ich sofort ``Rechte!'' rufe,
in der realen Welt nicht. Man kann nicht zweimal in denselben Fluss
eintreten! Und, wenn wir wissen, dass die M"unze in einer ein bisschen
ver"anderten Zukunft in Peters linken Hand war, ist es sehr
wahrscheinlich (ich betone, wahrscheinlich aber nicht n"otig!), dass
die M"unze in der realen Zukunft in derselben Hand ist!

HARRIS. Aber...

SMILLEY. Verstehen Sie noch nicht? Ok, hier ist ein anderes Beispiel,
obwohl kein gutes. Nehmen wir an, dass ich einen m"achtigen Rechner
habe, der, nachdem er die Situation (Gravitationverteilung, Biofelder,
usw.) analysiert hat, behauptet: ``Mit einer Wahrscheinlichkeit 99
Prozent ist die M"unze in der linken Hand.'' Ich h"ore zu und
gewinne. Wo ist die "Ubertretung der Ursache-Konsequenz Kette?
Verstehen Sie, es ist unm"oglich, die reale Zukunft zweimal zu
spielen, aber man kann die virtuelle durchlaufen und daraus etwas
lernen! Kurz gesagt, die virtuelle Zukunft ist das Modellieren der
realen Zukunft, wobei man dazu die Welt selbst benutzt!

HARRIS. Aber, wenn das erste Experiment nicht geschehen ist, wie
k"onnen Sie im zweiten den Antiziegel entdecken?

SMILLEY. Ich wiederhole: der Ziegel ist da, wenn das erste Experiment
in der virtuellen Zukunft gewesen, geschehen, eingetreten, vorgefallen
ist.

HARRIS. Seltsam... Na ja, ich werde mir das sp"ater "uberlegen.

SMILLEY. Es gibt nichts zu "uberlegen. Alles ist trivial w"ahrend die
M"oglichkeiten riesig sind. Und keine
Implikationswiderspr"uche. Keine!  "Unterstellen wir, dass ich heute
Abend ins Theater gehen will und ich zwischen ``Hamlet'' und
``Pigmallion''w"ahlen kann. Ich habe die beiden M"oglichkeiten
durchgespielt und es folgt aus meinen Berichten, dass ``Hamlet'' mir
besonders gefallen hat. Nehmen wir an, dass ein anderer Mensch
denselben Versuch macht und auch ``Hamlet'' w"ahlt. In der Realit"at
k"onnte es sein, dass er neben mir sitzt und sehr laut Popkorn mampft,
w"ahrend es ihn in meinem virtuellen Experiment gar nicht
gab. Selbstverst"andlich bereitet mir die Vorstellung kein Vergn"ugen.

HARRIS. Das hei"st, die virtuellen Varianten geben uns nur die Ann"aherung
der realen Zukunft?

SMILLEY. Ja. Gewiss kann man versuchen, in die ferne Zukunft zu
schauen. Zum Beispiel, w"urde ich in zehn Jahren reich, wenn ich nun
all mein Geld in die Aktien von ``Lokal Koordination, GmbH''
investiere?  Wir w"urden irgendeine Idee bekommen, aber in zehn Jahren
geschehen so viele Unterschiede, dass das Resultat nicht besser als
die Prophezeiung aus dem Kaffeesatz ist. Und es wird schlechter, weil die
Anderen den Imitator benutzen werden.

HARRIS. Wieso macht das das Ergebnis schlechter?

SMILLEY. Wie sonst? Wenn ich mit Ihnen Wappen-und-Zahlen spiele,
gewinne ich ganz trivial. Aber w"are mein Gegner genauso geschickt
beim Imitator wie ich, dann sind die Chancen
f"unfzig-f"unfzig. Symmetrie. Der Imitator funktioniert am besten,
wenn ich mit der toten Natur Wappen-und-Zahlen spiele. Soll man "Ol
hier oder da suchen? Soll man diesen Kranken nur mit Medikamenten
heilen oder eine Operation wagen?  Die M"oglichkeiten sind grenzenlos.

HARRIS. Ich kann es mir vorstellen. Wie ist es Ihnen gelungen, meinen
Trick mit der M"unze zu entdecken?

SMILLEY. Elementar. Der Imitator versucht immer alle Varianten zu
spielen. Und sie haben in beiden F"alle Ihre leere Hand rachgierig
ge"offnet und danach die H"ande schnell versteckt. Verzeihen Sie mir,
dass ich Ihnen dieses Vergn"ugen verdorben habe. He-he! Ich liefe dann
die Zweige, wo ich rufe ``Sie haben die M"unze verschluckt,'' ``Sie
ist in Ihrer Manschette,'' usw, bis die Version mit der Hosentasche
geklappt hat. Alles ist einfach.

HARRIS. Ja, alles ist einfach.

SMILLEY. Ein anderes Beispiel. Ich wollte gerne wissen, wer Sie sind,
warum Sie Ihre Zeit verlassen haben. Da Sie alles, was es bei Ihnen
gab, verloren haben. Und Sie wussten es oder haben es zumindestens
vermuten. Also, ich habe ein paar Versionen unseres Gespr"achs
durchlaufen, wo ich versuchte Sie dar"uber zu befragen. Leider ohne
Erfolg. Deswegen frage ich Sie nicht, weil Sie mir nichts antworten
w"urden.

HARRIS. Ja. Entschuldigen Sie, aber es ist schwierig f"ur mich
dar"uber zu sprechen.

SMILLEY. Wie ich gesagt habe. Ok, ich zeige Ihnen, wie es alles
praktisch aussieht. Man hat Ihnen ein Interface in Ihr Gehirn
eingesetzt. Das ist ein Ger"at, das Ihre Gedanken zum
Koordinationsrechner "ubertr"agt und vice versa. Regen Sie sich nicht
auf: ich habe auch so etwas. Der Rechner sieht Ihre Befehle f"ur den
Imitator in allen Gedanken. Da Sie noch keine Erfahrung haben, m"ussen
Sie, um ein Befehlj zu schicken, in Gedanken sagen: ``Imitator, ich
m"ochte das und dies.''  Danach sagt Ihnen der Rechner, was zu tun
ist.

HARRIS. Und wie lasse ich diesen Antiziegel hinlegen?

SMILLEY. Machen Sie sich keine Sorgen, das macht der
Koordinationsrechner. Obwohl es in der Wirklichkeit komplizierter ist.

HARRIS. Und was geschieht mit diesem, zweiten `ich'? 

SMILLEY. Nichts, es ist dem Zeitfluss entkommen und geht einen eigenen
Pfad entlang. Seien Sie ruhig; nachdem Sie sich von dem Zweiten
'verabschiedet' haben, k"onnen Sie sein Schicksal gar nicht (ich
betone, gar nicht) beeinflussen. Eigentlich streiten sich noch einige
Philosophen, ob es einen ausgew"alten Zeitfluss oder mehrere gibt. Die
theoretischen Physiker haben die M"oglichkeit des Zeitflusses justiert
und diese Theorie stimmt gut mit der Praxis ein. Der Imitator ist ein
starkes Argument, das die Theorie bekr"aftigt. Aber, eine Tatsache ist
immer eine Tatsache: virtuelle Welten existieren in der
Wirklichkeit. Entschuldigen Sie mein Wortspiel, he-he! Aber jetzt zur
Tat! Also, ich verberge die M"unze, w"ahrend Sie w"unschen:
``Imitator, ich m"ochte wissen, in welcher Hand die M"unze ist.''
Versuchen Sie es, seien Sie nicht so scheu.

HARRIS. Ok... Rechte?

SMILLEY. Genau! Was haben Sie empfunden?

HARRIS. Als ob meine (wie innere) Stimme mir ganz deutlich
``Rechte'' gesagt has. Ist sie die Stimmer des Imitators?

SMILLEY. Kann man sagen. Der Gedankenhersteller hat Ihnen die Antwort
des Imitators, der beide F"alle gespielt hat, geschickt.

HARRIS. Aber ich kann mich an so etwas, an solche Varianten, nicht
erinnern!

SMILLEY. Gewiss, weil sie nicht in der Wirklichkeit waren: sie sind
au"serfalls des Zeitflusses. M"ochten Sie es noch einmal probieren?

HARRIS. Gut, verstecken Sie bitte die M"unze... Rechte!

SMILLEY. Genau!

HARRIS. Muss ich jedes Mal wiederholen, ``Imitator, ich m"ochte das,''
``Imitator, ich m"ochte dies''?

SMILLEY. Bestimmt nicht. Sie werden bemerken, dass die Software am
Koordinationrechner ein kompliziertes und selbsverbesserndes System
ist. Bei mir, zum Beispiel, geschieht fast alles im
Unterbewusstsein. Ich schreie nie `Imitator!', doch ich wei"s,
dass der Rechner all meine W"unsche und Befehle in der richtigen Zeit
bearbeitet. Andererseits kann ich selbst nicht klar unterscheiden, wo
meine eigene Reaktionen und wo die externen Anleitungen sind; alles
ist so organisch verflochten und arbeitet zusammen. Je weiter, desto
mehr: der Imitator wird sogar Ihre unausgesprochenen W"unsche
voraussagen und erf"ullen. Mit der Zeit werden Sie merken, dass Sie im
Leben eine dauernde Gl"ucksstr"ahne haben; sogar bei solchen
Kleinigkeiten wie einen Bus, der genau an der Haltstelle erscheint,
wenn Sie ankommen.

HARRIS. Seltsam; Sie also benutzen bis heute Busse?

SMILLEY. In meiner Realit"at nicht; in Ihrer -- das h"angt von Ihnen
ab. Aber dar"uber sprechen wir sp"ater. Sie k"onnen sich gar nicht
vorstellen, was f"ur einen gro"sen Fortschritt das Leben seit der
Erfindung des Imitators gemacht hat. Zum Beispiel, die Rechner. Jetzt
kann man ein Programm in der Virtualit"at laufen lassen und, falls
keine Fehler (wie Stromausfall oder Speicherdefekt) auftreten, bekommt
man das Ergebnis gleich! Das das Resultat nur mit einer
Wahrscheinlichkeit von 99 Prozent wahr ist, schafft keine Probleme:
wieso sollte man nicht ein paar Kopien desselben Programms laufen
lassen?

HARRIS. Darf ich noch einmal versuchen zu raten, wo die M"unze legt?

SMILLEY. Sie haben mich irgendwie sonderbar angeschaut. Aber wieso nicht,
versuchen Sie!

HARRIS. Linke!

SMILLEY. Richtig!

HARRIS. Aber, warten Sie! Der Imitator hat gelogen! Er hat mir die Rechte
vorgeschlagen!

SMILLEY. Sie haben trotzdem die Linke gew"ahlt! Warum? Weil Sie sich
vorher entschlossen haben, gegenteil dessen zu tun, was der Imitator
geraten hat. Immer versuchen Sie so etwas zu schaffen! Und was ist
geschehen?

HARRIS. Aber, woher wusste der Imitator, dass er nicht richtige Seite
zeigen muss? A-a, jetzt verstehe ich!

SMILLEY. Also, woher?

HARRIS. Der Imitator hat zwei virtuelle Varianten durchlaufen, in
einer ratend die linke und in der anderen die rechte Hand zu
w"ahlen. Gewiss habe ich in beiden das Gegenteil gew"ahlt!

SMILLEY. Richtig! Also, welchen Rat bekommen Sie in der Wirklichkeit?
Solchen, der Ihnen die M"unze gibt, das hei"st, den falschen Rat! Sie
machen sehr schnell Fortschritt. Aber... W"are ich Sie, h"atte ich auf
den Imitator in allem geh"ort.

HARRIS. Ist das obligatorisch?

SMILLEY. Nein. Aber es ist besser f"ur Ihr Wohlergehen. Stimmen Sie
zu, dass, wenn Sie alle Ratschl"age des Imitators am genau erf"ullen,
dann wird der in ihnen enthaltene Zweck mit gr"o"serer
Wahrscheinlichkeit realisiert werden, als wenn Sie f"ur jeden
Ratschlag Ihre eigene Reaktion haben. Der Koordinationsrechner ist
eine wunderbare in virtuell-realer Zeit arbeitende Maschine, die,
au"ser, das sie all Ihre ������� erf"uhlt, sich auch um allgemeine
absolute Werte k"ummert, wie das Wohl f"ur alle und jeden. Wenn eine
virtuelle Variante Ihnen einen wahrscheinlichen Messerschlag in den
R"ucken gibt, gibt sich der Imitator alle M"uhe das zu vermeiden. Das
es keinen Dritten Weltkrieg gab, ist zweifellos wegen des Imitators.

HARRIS. K"onnte nicht jemand Imitatoren beherrschen, um zu... t"oten?

SMILLEY. Ausgeschlossen. All Ihre Gedanken, die in den Imitator gehen,
werden kontrolliert ob sie legal sind und den Rechtsgrundlagen
entsprechen. Kleineree Unfug kommt im Allgemeinen durch (wer ist ohne
S"unde?), aber die gr"o"ssere werden angehalten und eventuell von
Gericht gestellet, da das Gericht einen Prozess gegen Sie anfangen
k"onnte. "Uberdies werden Ihre Befehle f"ur den Imitator verschl"sselt
und in einen pers"onlichen Dossier gespeichert, der vom Gericht
benutzt werden kann, wenn der Hauptrat erlaubt, das pers"onliche File
zu decodieren. Man versucht das m"oglichst zu umgehen, aber manchmal
muss man es, wenn ein Prozess sich in hoffnungslosen Stillstand
bringt.

HARRIS. Aber man kann lernen, seine Gedanken zu kontrollieren. Ich
k"onnte, ein Messer in meinen Feind stechend, daran denken, was f"ur
einen Gefallen ich ihm tue und wie gl"ucklich ich ihn mache, wenn ich
ihm dis schwere Last des Lebens abnehme, da er von Vergn"ugen sogar
zittert und st"ohnt!

SMILLEY. Ja, aber eine Leiche ist immer eine Leiche, eine Tatsache. Es
wurde gesagt: ``Beurteile den Baum nach seinen Fr"uchten'' und die
Fr"uchte Ihrer Gedanken sind dem Imitator zug"anglich.

HARRIS. Ich stimme zu, wenn es eine Leiche gibt. Was, wenn ich
verschiedene kleine Sch"aden verursache und meinen Feind liebevoll
anschaue?

SMILLEY. Ja, aber ich habe schon gesagt, dass nicht alle Verbrechen
bestraft werden und, im Allgemeinen, nicht alle bestraft werden
sollen. Niemand versucht aus Ihnen einen idealen Menschen zu machen,
indem man Ihnen bei jedem b"osem Gedanken einen Elektroschock
versetzt. Und wieso? Aber wirklich, es ist eine sehr schwere Frage,
wo man die Linie zwischen Unfug und Verbrechen zeihen soll. Besonders,
da wir jedem das Recht garantieren, sein Privatleben zu sch"utzen. Im
Allgemeinen ist alles, was anderen Tot, schwere Verletzungen,
physische oder moralische Qualen verursachtt, verboten. Zum Beispiel,
wenn ich eine virtuelle Variante spiele, indem ich Sie w"urge, um
von Ihnen die Auskunft zu erzwingen, ist das nicht legal. Gewiss gibt
es viele Abstufungen im Gesetz. Alle Materialien sind auf Ihrem
Schreibtisch. Sei sollten sie gr"undlich studieren, bevor Sie den
Imitator ernsthaft benutzen.

HARRIS. Sie haben mir den Imitator gegeben. Haben Sie keine Angst, dass ich
etwas ganz Ungeheuers schaffe, einfach, weil ich keine
Erfahrung habe?

SMILLEY. Kaum m"oglich, besonders, da Sie ein eigenes lebendes Wesen in
Ihrer wirklichen Welt sind.

HARRIS. Entschuldigen Sie, sind Sie ein Roboter?

SMILLEY. Man kann sagen, ich bin die Projektion in Ihre Welt von einem
jetzigen Individuum.

HARRIS. Alles klar.. Na... [seufzt] Erz"ahlen Sie mir bitte, wie man
diesen Trick mit dem M"unzewerfen macht?

SMILLEY. In Prinzip ist alles einfach. Au"ser zu Ihren Gedanken hat
der Koordinationsrechner Zutritt zu Ihren Sinnen, wie Sehkraft oder
Ber"uhrungssinne. Ein Grund dafur ist, dass der Imitator so viel
Auskunft wie m"oglich "uber Ereignissen in der virtuellen Zukunft
braucht, um die Situation besser abzusch"atzen und die beste Variante
zu w"ahlen. Die pers"onliche Einsch"atzung ist nicht immer
objektiv. Man kann sich an ein Gedicht erinnern und die Wirklichkeit
vergessen. Deswegen ist es angeblich besser, wenn der
Koordinationsrecher die Situation direkt durch Ihre Sinnessensoren
untersuchen kann.

HARRIS. Und mir nach spionieren.

SMILLEY. Ja, wenn man das so nennt. Andererseits ist es auch
hilfreich, wenn der Imitator direkte Signale in Ihr Bewegungs- und
Nervensystem schicken kann. Der Trick mit dem M"unzwerfen ist ein
gutes Beispiel. Wenn ich m"ochte, dass die M"unze auf Wappen f"allt,
dann macht der Imitator eine Reihe von Experimenten, wo die M"unze von
verschiedenen H"ohen, mit verschiedenen Geschwindigkeiten und
Rotationsmomenten geworfen wird. Der Imitator w"ahlt die Parameter,
die Wappen bringen, und versucht in der Realit"at den Wurf mit
denselben Parameter zu wiederholen, was nur m"oglich ist, wenn er
direkt Ihre Muskeln kontrollieren kann. Das Wichtigste f"ur Sie ist,
es ihm nicht zu verwehren. Nach einigen "Ubungen werden Sie den Trick
schaffen k"onnen, was ich Ihnen jetzt empfehle zu tun. Lassen Sie uns
den Abschied nehmen. Ich hoffe, dass meine 'Vorlesung' nicht zu
m"uhsam f"ur Sie war.

HARRIS. Gewiss nicht. Aber, es ist ein bisschen...

SMILLEY. Seltsam. Ich verstehe. Morgen machen wir weiter. Auf
Wiedersehen.

HARRIS. Auf Wiedersehen.

***

HARRIS. [allein]

Ich m"ochte mir alles klarmachen. Ich habe ein kleines Experiment
geplant, in dem ich versuche alle meine Gedanke, Gef"uhle und
Erfahrungen  genaustens auf dieses Tonband aufzunehmen. Die
Alarmglocke ist genau auf sieben Uhr eingestellt. Jetzt ist es zehn
vor sieben. Jetzt gebe ich einen Befehl.

Imitator, ich m"ochte mich genau um sieben Uhr sehr gl"ucklich
f"uhlen!

Maggy?! Kann nicht sein! Wieso?...

Aber ich habe es vergessen. Versuche mich zusammenzunehmen. Keine
Gef"uhle. Also. Der Imitator hat mir das Folgende gesagt: ``Kurz nach
Ihrem Verschwinden hat Maggy Bland, Ihre Laborantin, versucht, Ihr
Experiment zu wiederholen, aber wurde durch einen nicht richtigen
Stromeinsatz get"otet.''

Verstehe nichts... Ich muss "uberlegen. Warum wurde mir das gesagt?
Der Imitator sollte ein paar Varianten spielen und eine w"ahlen, wo
ich punkt sieben Uhr sehr gl"ucklich bin. Es ist unwahrscheinlich,
dass diese Nachricht mich sehr freuen wird. Gewiss sind alle, die
damals gelebt haben, schon lange tot, aber als etwas "uber
Zwanzig-j"ahrige f"ur nichts zu sterben? So dumm. Vielleicht hat der
Imitator mich nicht verstanden oder wei"s nicht, was `gl"ucklich'
bedeutet. Vielleicht war ich in keiner Variante gl"ucklich. K"onnte
sein.

Warte. Der Imitator kann wohl l"ugen, wie zum Beispiel mit der
M"unze. Aber andererseits konnte er sich dies nicht allein ausdenken;
irgendwoher muss er die Information genommen haben. Aus mir --
unwahrscheinlich: ich habe mich an sie nicht einmal erinnert. Aus
meinem Unterbewusstsein?  Mm... Aber auf jeden Fall bleibt die Frage,
ob Maggy wirklich auf diese Weise gestorben ist. Schade, man kann das
nicht "uberpr"ufen!  Doch... ihr Name k"onnte in irgendeiner damaligen
Zeitung in der Sterbeliste sein. Morgen muss ich unbedingt nach den alten
Zeitungen fragen.

Ok. Wenn das eine L"uge ist, dann eine scheu"sliche und unn"otige
L"uge. Wenn das keine... Dann, warum hat Maggy es getan? Um des Ruhms
willen?  Nein, Quatsch: zu dieser Zeit waren solche Experimente keine
Neuigkeit. Dazu ein gro"ses Risiko. Selbstmord? Aber warum genau dann
und genau auf diese Weise? Und warum wurde ich eben benachrichtigt?

K"onnte es sein... k"onnte es sein, dass sie mich so geliebt hat, dass
sieversuchte, mich zu folgen ? Aber ich kann mich an keine ihrer
Flirtversuche erinnern. Au"ser, wenn man es einen besonderen Flirt
nennt, dass sie immer ohne Umstande bereit war, jede schmutzige Arbeit
zu machen oder mir mit den Experimenten abends nach der Arbeitszeit zu
helfen. Ein weinerliches M"adchen, dessen Kopf voll von seltsamen
Idealen war. Sie hat sich immer h"asslich gekleidet, die Kleider sind
an ihr wie Kattun gehangen. Blondes Haar. An ihre Augenfarbe kann ich
mich nicht erinnern, auch wenn ich es stundenlang versuchen
w"urde. Trotzdem hat sie, wie eine Laborantin, immer sorgfaltig
bearbeitet, und war im allgemeinen klug.

Ich erinnere mich, als ich Sie einmal gefragt habe, warum sie nicht
versucht hatte, irgendwo anders den Doktortitel zu anwerben. Ja, ich
erinnere mich wohl, sie ist gleich matt geworden und hat etwas
unklares geantwortet.

Doch hat sie richtig vermutet, wohin ich verschwunden war! Jetzt
scheine ich dieses seltsame Gespr"ach zu verstehen, wenn sie mich
schwankend gebeten hat, diese Experimente aufzugeben. Das Gespr"ach
war noch nicht beendet, da wurde sie rot und ist weinend weglaufen
(ihr Stil). Aber warum, warum konnte sie mir nichts genaues und
direktes sagen?! Braucht man viel Mut dazu? Andererseits, wie konnte
sie sich f"ur solchen Schritt entschlie"sen? Sie war so charakterlos
-- wie sie einmal untr"ostlich geweint hat, als ich sie f"ur das
zerbrochene Mikroskop beschimpft habe. Ein Ozean von Tr"anen.

Liebe? Zu mir? Ich bin ein bekannter Wissenschaftler, sehe zwar
imponierend aus, aber all meine unangenehmen Techtelmechtel im
Institut oder drau"sen, die sie selbst gesehen oder von Institutsdamen
geh"ort hat. Worauf konnte sie hoffen? Ein Liebling f"ur einen Monat?
Das sollte sie gewiss verstehen. Aber andererseits gibt es so viel
Irrationales bei Frauen.

Doch, echt! Das alles passt genau zusamen, St"uck f"ur St"uck, alle
Einzelheiten zueinander. Wie konnte ich so ein Esel sein und die ganze
Zeit nicht merken? Einfach in Ruhe "uberdenken, alle Tatsachen sind
gleich an der Oberfl"ache.

Wirklich, wenn jemand mich liebte und mir wahrhaftig treu war, dann
war es wahrscheinlich nur sie... Was folgt daraus? Dass ich an ihrem
Tod schuldig bin? An ihrem Tod, die eine unschuldige Magd war, die so
viel gelitten hat, nur weil sie mich liebte?

[Alarmsignal klingt.]

ERZ"AHLER Aber dies ist nicht geschehen. Etwa zehn Minuten vor sieben
wird die Zeit durch einen anderen Pfad laufen. Professor Harris wird
die M"unze werfen und jedes Mal laut lacheln, wenn Wappen
erscheint. Und Wappen wird jedes Mal erscheinen... An diesem Abend
wird Maggys Name nicht genannt werden. Wahrscheinlich wird der
Professor nie das Schicksal seiner unscheinbaren Laborantin erfahren, weil
niemand es ihm erz"ahlen wird: der Imitator ist eine wunderbare
Maschine, die uns vor Tod, Schmerz oder unn"otigen Sorgen sch"utzt.



\vspace{30pt}\noindent Aus Russich vom Autor "ubersetzt. Der Autor dankt
Sabine Giese f"ur mehrere Sprachkorrektionen.

\end{document}