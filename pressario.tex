LOG-JOURNAL
LINUX ON EMACHINES M5305 NOTEBOOK

Here are the changes to the default system/configuration which I found
useful.


11/13
installed Suse 9.0. Many useful packages were not selected by default:
tetex, xfig, rxvt, emacs+, lzo (needed by mplayer), findutils-locate


11/15
To be compatible with previous versions created /media/flash 
added to 
/etc/fstab:
/dev/sda1            /media/Flash         vfat       rw,noauto,user        0 0

renamed /media/floppy into /media/Floppy

tried to configure display using sax2. Manually chose freq 30-50 & 50-70 in
the monitor section and 1280x800 in resolution and it worked! The current
XF86Config is copied here as XF86Config.031115

istalled mplayer (and required packages from packman). works! (although
not smooth - is something eating machine's resources up?

/etc/inittab
# what to do when CTRL-ALT-DEL is pressed; let us set to halt
ca::ctrlaltdel:/sbin/shutdown -h -t 4 now

11/16
 installed findutils-locate. This package puts some files in /etc/cron.daily
which are run (accordingly to /etc/crontab at 4:14am each time). Hopefully,
updatedb won't start 15min after each startup.


11/22
 tried to use PCMCIA card for CF. In /var/log/messages found a reference to
hde1. Added as root to /etc/fstab
/dev/hde1            /media/PcmciaFlash/  vfat       rw,noauto,user        0 0


11/24-26
the linux system crashed: grub took 10 sec to load and then I was getting
the message "Kernel panic: journal points to accessible data. (Previous when I
was shutting down the computer it hanged up on shutting down pcmcia so I just
turned it off. Florin suggested that fs is corrupt and that I better reistall
all. However, I decided to try rescue. I read Suse82 Adm Guide p.284 and
at the #-prompt I typed
 reiserfs --check /dev/hda5
 It returned "bad superblock" so I had to run with option --update-sb. At
first it seemed to fix everything but then mouse stopped working and other
errors, so I guessed Florin was right. (As usually.)

Also he advised to migrate to ext3, which is more reliable in his
opinion. Reformatted hda5 as ext3 during install then copied everything from
hda7 (/home) to hda5, unmounted hda7 and ran 
 mkfs.ext3 -c -c /dev/hda7
 (The double -c means a complete write-read test for bad blocks.) Then
replaced reiserfs by ext3 in fstab. Looks OK.

11/28
 commented out the following line in /etc/crontab
-*/15 * * * *   root  test -x /usr/lib/cron/run-crons && /usr/lib/cron/run-crons >/dev/null 2>&1

 reason: it looks that run-crons is responsible for running e.g. cron.daily if
they are more than 24 hours old. In practice this means that in 15 min ater
start power-on the machine may be busy performing these tasks. Stupid.

>>Later comment: this was bad soltion, see the entry for 25/5


In /usr/lib/cron/run-crons I found a (potentially) useful code for checking
whether the laptop is plugged into power socket:

if test -x /usr/bin/apm ; then
    case "`/usr/bin/apm 2> /dev/null`" in
      "AC off-line"*)
        # Laptop is offline
        exit 0
      ;;
    esac
fi

Unfortunately, when I run it on EM the output is "No APM support in kernel". 
Apparently I should read the acpi documentation (not apm).

11/29 installed powertweak and powertweak-extra: it seems that one can pass
many options to kernel using it. Decided better not to play with it.

yast ->  system -> profile manager;
enabled scpm
saved current configuration as default;
created a new configuration "home" based on default;
Chose "home" to be the current profile.

Yast -> system -> runlevel
 disabled cupsd, isdn, nscd, postfix, smbfs,
 wanted to disable network but yast warmed me that then I have to disable
cpufreq, xdm, and many other useful processes dependent on network. So I left
network running.

Having rebooted the laptop it seems work (no cupsd, etc)!

When I typed "scpm switch default" as the root the computer started cups,
postfix, and restarted network. When I ran "scpm switch home" it stopped
postfix and cups. Wonderful!

I also discovered how to check whether the computer is on AC or battery:
read file /proc/acpi/ac_adapter/AC/state 
 When on battery it consists of the line
state:                   off-line
 Otherwise it contains
state:                   on-line
 Also, file /proc/acpi/battery/BAT0/state contains various info on the current
state of the battery (such as remaining capacity, etc). Neat.

Now I try to make the gyration wireless usb mouse work. First, here is the old
section from XF86Config describing the built-in mouse:
Section "InputDevice"
  Driver       "mouse"
  Identifier   "Mouse[1]"
  Option       "Buttons" "5"
  Option       "Device" "/dev/mouse"
  Option       "Name" "Autodetection"
  Option       "Protocol" "imps/2"
  Option       "Vendor" "Sysp"
  Option       "ZAxisMapping" "4 5"
EndSection

The new setting which works with the wireless mouse is
Section "InputDevice"
  Driver       "mouse"
  Identifier   "Mouse[1]"
  Option       "ButtonNumber" "3"
  Option       "Buttons" "5"
  Option       "Device" "/dev/mouse"
  Option       "Name" "Autodetection"
  Option       "Protocol" "PS/2"
  Option       "Vendor" "Sysp"
  Option       "ZAxisMapping" "4 5"
EndSection

Copied the new config as XF86Config.031129 here. After a few errors and trials
I have found a solution! One trick is that yast2 keeps /dev/mouse a link
to the actuall device. For build-in it should be /dev/psaux ; for usb it is
/dev/input/mice. Hence, in XF86Config we include both sections (with different
Identifiers of course):

Section "InputDevice"
  Driver       "mouse"
  Identifier   "Mouse[1]"
  Option       "Buttons" "5"
  Option       "Device" "/dev/psaux"
  Option       "Name" "Autodetection"
  Option       "Protocol" "imps/2"
  Option       "Vendor" "Sysp"
  Option       "ZAxisMapping" "4 5"
EndSection

Section "InputDevice"
  Driver       "mouse"
  Identifier   "Mouse[2]"
  Option       "ButtonNumber" "3"
  Option       "Buttons" "5"
  Option       "Device" "/dev/input/mice"
  Option       "Name" "Autodetection"
  Option       "Protocol" "PS/2"
  Option       "Vendor" "Sysp"
  Option       "ZAxisMapping" "4 5"
EndSection

Finally, in the Section "ServerLayout" one can describe two mice as follows
(I found this reading man XF86Config)

Section "ServerLayout"
  Identifier   "Layout[all]"
  InputDevice  "Keyboard[0]" "CoreKeyboard"
  InputDevice  "Mouse[1]" "CorePointer"
  InputDevice  "Mouse[2]" "SendCoreEvents"
  Option       "Clone" "off"
  Option       "Xinerama" "off"
  Screen       "Screen[0]"
EndSection

This works but there are some unpleasantries: when you run sax2 on the
modified file, the new XF86Config may be not working. For
example, when I tried this, sax2 mixed keyboard and mouse settings, like
 Section "InputDevice"
   Driver "Keyboard"
   ...
   Option "Protocol" "imps/2"

11/30
 installed Maple 9. It seems that Maple installs its own Java runtime
environment: possible colisions with Suse Java config :-?

12/6
 KDE Control Center -> System Administration -> Login Manager -> Convevience
 Chose "Preselect User"=Previous, Focus Password=yes
 (Less typing when loggin in :0) )

 installed MPlayer and necessary files (from packman). The gmplayer would not
start complaining that there is nor read access to /dev/rtc 
Changed the group of this file to users and added g+r access. gmplayer works.

12/7
 I have finally found a way to rotate the files from the digital camera.
First I installed jhead (file jhead-1.5.0.tar.gz here) which, if envoked
with option -verbose output all JPEG header content, including the
orientation. Here is my script (clumsy but seems to work):

echo "Rotating files which need rotation (lossless)"
for i in *.jpg
 do
 ORIENT=`jhead -v $i | grep "Orientation" | sed "s/^ *Orientation = //"`
 echo "$ORIENT"
 if [ "$ORIENT" == "6" ]
 then
  echo "We have to rotate $i"
#  use the following line if wishing to preserve EXIF information
#  jhead -cmd "cp $i $i.JpG; jpegtran -rotate 90 -trim $i.JpG > $i; rm $i.JpG"  $i
  cp $i $i.JpG; jpegtran -rotate 90 -trim $i.JpG > $i; rm $i.JpG
 else
  if [ "$ORIENT" == "8" ]
  then
   echo "We have to rotate $i"
#  use the following line if wishing to preserve EXIF information
#  jhead -cmd "cp $i $i.JpG; jpegtran -rotate 270 -trim $i.JpG > $i; rm $i.JpG"  $i
 cp $i $i.JpG; jpegtran -rotate 270 -trim $i.JpG > $i; rm $i.JpG
   # else
   # echo "No rotation"
  fi
 fi
done

One hint: if the pixel dimentions of a jpeg file are not multiple of 8, then
the lossless rotation is not possible; the option -trim eliminates the
onconverted strip of pixels at margins.

Another hint:
# If you want to preserve EXIF information
jhead -cmd 'jpegtran -rotate 90 -outfile &o &i' my-pic.jpg


Also, the author of jhead says that mogrify -quality 80% decreases the image
size by 1/2 without losing quality. Unfortunately, for my settings (Large size
(4M) and normal compression) this seems to get only about 15% improvement. Not
worth bothering.


Added the following lines to ~/.profile

export PATH=.:$PATH
export PS1="\w> "
export TEXINPUTS=$TEXINPUTS:$(HOME)/tex/input
export BIBINPUTS=$(HOME)/tex/bib

The bash man-page seems to specify .bash_profile as the proper place to
one-time settings. But when I looked at the standard SuSe .bashrc I found the
following text

# NOTE: It is recommended to make language settings in ~/.profile rather than
# here, since multilingual X sessions would not work properly if LANG is over-
# ridden in every subshell.

However, it seems that $(HOME) was replaced by the empty string. $HOME works.
Also, the prompt setting was overriden by the system-wide setting, so I moved
the PS-stuff in .bashrc (Also I replaced \w by \W - only the last element of
the full pwd is displayed.)

Problem: I want to run some files before starting a WM but also I want to be
able to choose a WM from xdm. Solution: Here is my .xinitrc

xset fp+ $HOME/Fonts/ukrainian/
xset fp+ $HOME/Fonts/cyrillic_w
xset +fp $HOME/Fonts/cyrillic_k
xmodmap $HOME/bin/jcuken-koi8u-xrus.xmm
$HOME/bin/xrus

if [ "$WINDOWMANAGER" == "fvwm2" ]
then
 exec fvwm2
else
 startkde
fi

1/1 Happy New Year 2004! :-)

1/3
xboard would not start: some fonts are missing. The following works
xboard -clockFont "-adobe-courier-*-r-*" &

Try to put the following into ~/.Xdefaults file but it doesn't work...
Xboard*Font "-adobe-courier-*-r-*" Hm...

18/5
put "unalias l" into .bashrc (otherwise l is ls -l)

25/5 installed jpeg tools from http://www.ijg.org/

25/5 
 Just realised that what I did to /etc/cron.tab stopped doing any cron jobs!

Dirt solution (to prevent running update_db 15 min after laptop is switched on):
 i) copied /usr/lib/cron/run-crons to /usr/lib/cron/run-crons-no-daily
 ii) removed daily there
 iii) added the following lines to /etc/cron:
-*/15 * * * *   root  test -x /usr/lib/cron/run-crons-no-daily && /usr/lib/cron/run-crons-no-daily >/dev/null 2>&1
15 4 * * *   root  test -x /usr/lib/cron/run-crons && /usr/lib/cron/run-crons >/dev/null 2>&1

2/7
 Finally, I found a way to configure the touchpad tapping! I used the
synaptics driver (web-page http://w1.894.telia.com/~u89404340/touchpad/)

Installed version 0.13.3. I added to XF86Config

Section "InputDevice"
  Identifier    "Synaptics Mouse"
  Driver        "synaptics"
  Option        "Device"        "/dev/psaux"
  Option        "Protocol"      "auto-dev"
  Option        "LeftEdge"      "1700"
  Option        "RightEdge"     "5750"
  Option        "TopEdge"       "1700"
  Option        "BottomEdge"    "4200"
  Option        "FingerLow"     "25"
  Option        "FingerHigh"    "30"
  Option        "MaxTapTime"    "0"
  Option        "MaxTapMove"    "220"
  Option        "VertScrollDelta" "100"
  Option        "MinSpeed"      "0.06"
  Option        "MaxSpeed"      "0.12"
  Option        "AccelFactor" "0.0010"
  Option        "SHMConfig"     "off"
#  Option       "Repeater"      "/dev/ps2mouse"
EndSection

The option MaxTapTime=0 disables tapping. Also in Section
"ServerLayout" I made the following changes:

  InputDevice  "Synaptics Mouse" "CorePointer"
#  InputDevice  "Mouse[1]" "CorePointer"

Started X from a virtual console with 

startx -- :1

All works as desired :-)

My changes deviating from the default: 
  Option        "SHMConfig"     "off"
(on allows changing parameters on fly but may be not very secure, according
to the INSTALL file in the distribution).

  Option        "RightEdge"     "5800"
 (ajusted the right border between mouse movement and scrolling to suit my
touchpad better)

I also noticed that the package has the following program 
(another possible solution to tapping):

syndaemon  - a program that monitors keyboard activity and
disables the touchpad when the keyboard is being used.

6/7
Added the following to /etc/init.d/boot.local 

#OP adjusting clock drift (parameters obtained by running adjtimex -c)
/usr/sbin/adjtimex --tick 1100  --freq 0

Explanations: "adjtimex -c" recommends 1124 but unfortunately 1100 is 
the highest. Still something is very fishy: with -tick 1100, the system
runs too fast in comparison with my watch (+30sec in 5 min). But, adjtimex -c
returns the same offset even if I run xntpd. Hm?... 

Well, I'll just experiment and by error and trial adjust the clock


Here is the /etc/ntp.conf file (created by yast):

server 127.127.1.0
fudge 127.127.1.0 stratum 10
server 217.153.69.35
driftfile /var/lib/ntp/drift/ntp.drift
logfile /var/log/ntp

I decided to experiment using the following config:
server pool.ntp.org                                                            
server pool.ntp.org
server pool.ntp.org 
                                                      
server 127.127.1.0 
fudge 127.127.1.0 stratum 10
driftfile /var/lib/ntp/drift/ntp.drift
logfile /var/log/ntp

11/7
In /etc/modules.conf, in the line
options snd-ali5451 snd_enable=1 snd_index=0 snd_pcm_channels=32 snd_spdif=0
removed the last option which just causes a warning message during startup


13/8
Got a wireless card from Florin. It got autodetected.

According to a web-site I changed the following via yast2: Wireless
options (here is how they appear in
/etc/sysconfig/network/ifcfg-wlan-pcmci

WIRELESS_ESSID='CMU'
WIRELESS_KEY=''
WIRELESS_MODE='Managed'

Unfortunately, the default gateway did not appear. So Florin add it
manually:

# route add default gw 128.237.224.1

This changed the file /etc/sysconfig/network/routes to one-line
default 128.237.224.1 - - 

It seems I have to type  route add default gw 128.237.224.1 
after start-up :-((

18/8 Florin has figured out the reason: the interface eth0 was a kind
of primary and was responsible for gateway. Therefore, in the file
/etc/sysconfig/network/ifcfg-eth0 he replaced 
 STARTMODE='onboot' by STARTMODE=''

The result is that eth0 is activated now. If later, I want to use it,
I have to change it back.

Later in my office I discovered that the network/gateway comes up but
nameservers do not property configure. So, in yast2 I just entered the
manual configuration: name servers 128.2.1.11 and 128.2.1.10 (plus
domain search cmu.edu and math.cmu.edu)



*********************** The fun(?) continues **************************

8/28/2005 I have bought new notebook (Compaq Presario V2312US). 
Installed Suse 9.3. Many things still needs to be fixed.

8/28 To switch off tap=click behavior of the touchpad, added the line
  Option "MaxTapTime=0"
into section with Driver "synaptics". For more info, see
web.telia.com/~u89404340/touchpad/


8/29 soundcard did not work at first but after the first online update, it got 
properly detected and configured

8/30
 To change the login manager:
 /etc/rc.d/xdm runs /etc/sysconfig/displaymanager which in turns sets us
variable $DISPLAYMANAGER to "gdm" or "kdm".

8/31
 After the fglrx driver from ita was installed, synaptics stopped responding 
to Option "MaxTapTime=0". Still do not know why. Semigood solutions: upgraded
synaptics to the current version 0.14 which comes with synclient. Then in
gnome-control-center -> sessions -> startup programs I entered
 synclient MaxTapTime=0




